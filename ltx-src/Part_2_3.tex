
% This LaTeX was auto-generated from an M-file by MATLAB.
% To make changes, update the M-file and republish this document.

%\documentclass{standalone}
%\usepackage{graphicx}
%\usepackage{color}
%\usepackage{standalone} 

%\sloppy
%\definecolor{lightgray}{gray}{0.5}
%\setlength{\parindent}{0pt}

%\begin{document}

    
    \begin{par}

We use the function \texttt{getH(sigma)}
to compute the matrices. The source code after this part.
for $\sigma_1 = 1$ and $\sigma_2 = 2$.

\end{par} \vspace{1em}
\begin{verbatim}
H1 = getH(1)
H2 = getH(2)
%Normalisation
%Calculate sums
s1 = sum(H1(:))
s2 = sum(H2(:))
%Multiply with the inverse
H1 = 1/s1 * H1;
H2 = 1/s2 * H2;
%Control
s1 = sum(H1(:))
s2 = sum(H2(:))
% Visualisation with mesh
x = -2:2;
[X,Y] = meshgrid(x,x); %create grid
mesh(X,Y,H1);
figure;
mesh(X,Y,H2);
\end{verbatim}

        \color{lightgray} \begin{verbatim}
H1 =

    0.0029    0.0131    0.0215    0.0131    0.0029
    0.0131    0.0585    0.0965    0.0585    0.0131
    0.0215    0.0965    0.1592    0.0965    0.0215
    0.0131    0.0585    0.0965    0.0585    0.0131
    0.0029    0.0131    0.0215    0.0131    0.0029


H2 =

    0.0146    0.0213    0.0241    0.0213    0.0146
    0.0213    0.0310    0.0351    0.0310    0.0213
    0.0241    0.0351    0.0398    0.0351    0.0241
    0.0213    0.0310    0.0351    0.0310    0.0213
    0.0146    0.0213    0.0241    0.0213    0.0146


s1 =

    0.9818


s2 =

    0.6297


s1 =

     1


s2 =

    1.0000

\end{verbatim} \color{black}
    
\includegraphics [width=4in]{Part_2_3_01.eps}

\includegraphics [width=4in]{Part_2_3_02.eps}
\begin{par}

We see, that both matrices are invariant under $90^{\circ}$ rotations,
reflection and transposing. In $H_1$, the center has a higher value
compared to the other points. \\ \\ Now we import the Picture ntugn.jpg
from our image folder and view it. To get a better result, we also do a
contrast stretching.

\end{par} \vspace{1em}
\begin{verbatim}
P = imread('images/ntugn.jpg');
imshow(P,[]);
\end{verbatim}

\includegraphics [width=4in]{Part_2_3_03.eps}
\begin{par}

We see that there is a lot of noise. For example in the sky should
adjacent neighbours have a similar value. This is not the case here. So
we apply our Gaussian filters with zero padding.

\end{par} \vspace{1em}
\begin{verbatim}
P1 = conv2(double(P),double(H1),'same');
P2 = conv2(double(P),double(H2),'same');
% Get back to uint8
P1 = uint8(P1);
P2 = uint8(P2);
% Show results, also with contrast stretching
imshow(P1,[]);
figure;
imshow(P2,[]);
\end{verbatim}

\includegraphics [width=4in]{Part_2_3_04.eps}

\includegraphics [width=4in]{Part_2_3_05.eps}
\begin{par}

We obtained two pictures with less noise. If $\sigma$ is higher more
noise is cancelled, but the image is also more blurred. Now we apply the
same filters to the picture ntu-sp.

\end{par} \vspace{1em}
\begin{verbatim}
Q = imread('images/ntusp.jpg');
imshow(Q,[]);
\end{verbatim}

\includegraphics [width=4in]{Part_2_3_06.eps}
\begin{par}

We see, that the noise is different. In this setting the noise is very
bright. Now we apply our filters

\end{par} \vspace{1em}
\begin{verbatim}
Q1 = conv2(double(Q),double(H1),'same');
Q2 = conv2(double(Q),double(H2),'same');
% Get back to uint8
Q1 = uint8(Q1);
Q2 = uint8(Q2);
% Show results, also with contrast stretching
imshow(Q1,[]);
figure;
imshow(Q2,[]);
\end{verbatim}

\includegraphics [width=4in]{Part_2_3_07.eps}

\includegraphics [width=4in]{Part_2_3_08.eps}
\begin{par}

In this case, Gaussian filtering is not the best choice. In the first
picture the noise value was close to the adjacent pixel values. In this case Gaussian filtering can not work so well, as the
noise value is generally far away from the values of the adjacent pixels.
So we see, that the brightness of the noise pixels decrease, but the
spots get larger, as the noise also affects the adjacent pixels more. In this
case, median filtering might be better.

\end{par} \vspace{1em}



% 	\end{document}
    
