
% This LaTeX was auto-generated from an M-file by MATLAB.
% To make changes, update the M-file and republish this document.

%\documentclass{standalone}
%\usepackage{graphicx}
%\usepackage{color}
%\usepackage{standalone} 

%\sloppy
%\definecolor{lightgray}{gray}{0.5}
%\setlength{\parindent}{0pt}

%\begin{document}

    
    \begin{par}

All images are stored in a separate folder called 'images'. This leads to
a tidy workspace. But we have to add the name of the folder to the image name.

\end{par} \vspace{1em}
\begin{verbatim}
Pc = imread('images/mrttrainbland.jpg');
\end{verbatim}
\begin{par}

The command \texttt{whos} gives a short summary of the matrix properties.

\end{par} \vspace{1em}
\begin{verbatim}
whos Pc
\end{verbatim}

        \color{lightgray} \begin{verbatim}  Name        Size                Bytes  Class    Attributes

  Pc        320x443x3            425280  uint8              

\end{verbatim} \color{black}
    \begin{par}

We see, that the matrix has RGB values. We see this because the third dimension is 3. To convert it to a matrix with
grey-scale values, we use the method \texttt{rgb2gray}

\end{par} \vspace{1em}
\begin{verbatim}
P =rgb2gray(Pc);
whos P
\end{verbatim}

        \color{lightgray} \begin{verbatim}  Name        Size              Bytes  Class    Attributes

  P         320x443            141760  uint8              

\end{verbatim} \color{black}
    \begin{par}

The command \texttt{imshow} shows the picture in a new figure.

\end{par} \vspace{1em}
\begin{verbatim}
imshow(P)
\end{verbatim}

\includegraphics [width=4in]{Part_2_1_01.eps}
\begin{par}

We find the minimum $\min_P$ and maximum $\max_P$ intensities with the following:

\end{par} \vspace{1em}
\begin{verbatim}
min_P = min(P(:)),max_P = max(P(:))
\end{verbatim}

        \color{lightgray} \begin{verbatim}
min_P =

   13


max_P =

  204

\end{verbatim} \color{black}
    \begin{par}

We want a picture $P_1$, such that $\min_{P_1} = 0 $ and $\max_{P_1}
= 255$. To reach this, we subtract at first $\min_P$ from every value and
multiply the result with $\frac{\max_{P_1}}{\max_P - \min_P}$

\end{par} \vspace{1em}
\begin{verbatim}
P1 =  imsubtract(P,13);
P1 = immultiply(P1, (255/(204-13)));
\end{verbatim}
\begin{par}

Note that this works only if $\max_P \not = \min_P$. We see, that the
minimum and maximum intensity values are correct and that the picture was enhanced.

\end{par} \vspace{1em}
\begin{verbatim}
min_P1 = min(P1(:)), max_P1 = max(P1(:))
imshow(P1)
\end{verbatim}

        \color{lightgray} \begin{verbatim}
min_P1 =

    0


max_P1 =

  255

\end{verbatim} \color{black}
    
\includegraphics [width=4in]{Part_2_1_02.eps}
\begin{par}

Using the method \texttt{imshow(P,[])} does the contrast stretching
automatically. So we should see now the same picture as above.

\end{par} \vspace{1em}
\begin{verbatim}
imshow(P,[])
\end{verbatim}

\includegraphics [width=4in]{Part_2_1_03.eps}



% 	\end{document}
    
