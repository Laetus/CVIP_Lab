
% This LaTeX was auto-generated from an M-file by MATLAB.
% To make changes, update the M-file and republish this document.

%\documentclass{standalone}
%\usepackage{graphicx}
%\usepackage{color}
%\usepackage{standalone}�

%\sloppy
%\definecolor{lightgray}{gray}{0.5}
%\setlength{\parindent}{0pt}

%\begin{document}

    
    \begin{par}

In a first step, we load the image,convert it into a a grey level image
and display it.

\end{par} \vspace{1em}
\begin{verbatim}
close all
clear all
P = imread('images/maccropped.jpg');
P = rgb2gray(P);
imshow(P);
\end{verbatim}

\includegraphics [width=4in]{Part_3_1_01.eps}
\begin{par}

We want to convolute the Sobel mask with the image, therefore the vertical matrix
must have the following form: $$ V =  \begin{pmatrix} 1 & 0 &-1 \\ 2 & 0 & -2
\\1 & 0 & -1 \end{pmatrix}$$ The horizontal Sobel filter is: $$ H = V^T
$$ If a edge is not vertical nor horizontal, we can not detect it. We see
this, if we at the edge which separates street and grass. To detect edges with an angle of 45${}^\circ$, the following matrices are suitable.
$$D_- = \begin{pmatrix} 0 & 1 & 2 \\ -1 & 0 & 1 \\ -2 & -1 & 0\end{pmatrix} \ \text{and} \ D_+ =  \begin{pmatrix} -2 & -1 & 0 \\ -1 & 0 & 1 \\ 0 & 1 & 2 \end{pmatrix} $$
The results of both filters are displayed below.

\end{par} \vspace{1em}
\begin{verbatim}
vert = [1 0 -1 ;
        2 0 -2 ;
        1 0 -1];
hori = vert';
%Apply convolution
Pv = conv2(double(P),double(vert),'same');
Ph = conv2(double(P),double(hori),'same');
imshow(uint8(Pv))
title('Vertical  Sobel')
figure;
imshow(uint8(Ph))
title('Horizontal Sobel')
\end{verbatim}

\includegraphics [width=4in]{Part_3_1_02.eps}

\includegraphics [width=4in]{Part_3_1_03.eps}
\begin{par}

Now we add and square the results \texttt{Pv} and \texttt{Ph}. In this
case there are two main reasons to square the sum. Firstly squaring
eliminates negative values. Secondly the difference between the values
are amplified.

\end{par} \vspace{1em}
\begin{verbatim}
Ps = imadd(immultiply(Pv, Pv),immultiply(Ph, Ph));
Ps = sqrt(Ps);
%show result
imshow(uint8(Ps));
\end{verbatim}

\includegraphics [width=4in]{Part_3_1_04.eps}
\begin{par}

Now we apply different thresholds. I chose the following thresholds: $$
\tau \in \{32, 64, 128,180,200,245\} $$ If the threshold is chosen too low, we have to
much information. If the threshold is chosen too high, important
information and details are lost.

\end{par} \vspace{1em}
\begin{verbatim}
E32 = Ps > 32;
E64 = Ps > 64;
E128 = Ps > 128;
E180 = Ps > 180;
E200 = Ps > 200;
E245 = Ps > 245;
subplot(2,3,1)
imshow(E32);
title('\tau = 32')
subplot(2,3,2)
imshow(E64);
title('\tau = 64')
subplot(2,3,3)
imshow(E128);
title('\tau = 128')
subplot(2,3,4)
imshow(E180);
title('\tau = 180')
subplot(2,3,5)
imshow(E200);
title('\tau = 200')
subplot(2,3,6)
imshow(E245);
title('\tau = 245')
\end{verbatim}

\includegraphics [width=4in]{Part_3_1_05.eps}
\begin{par}

The parameter $\sigma$ is the standard deviation of the Gaussian filter.
As we have learned in class, this parameter decides how much influence
the adjacent pixels have. In one of the previous experiments was shown,
that a higher $ \sigma$ value leads to more noise cancellation, but small
details can disappear and the images becomes more blurred. In this case
the results are similar. If $\sigma$ is small we see a lot of edges and
also noise, e.g. in the grass. If $\sigma$ is higher, we might have lost
some edges, but more noise is reduced.

\end{par} \vspace{1em}
\begin{verbatim}
subplot(2,3,1)
E1 = edge(P,'canny', [.04 .1], 1);
imshow(E1);
title('\sigma = 1');
subplot(2,3,2)
E2 = edge(P,'canny', [.04 .1], 2);
imshow(E2);
title('\sigma = 2');
subplot(2,3,3)
E2_5 = edge(P, 'canny', [.04 .1] , 2.5);
imshow(E2_5);
title('\sigma = 2.5');
subplot(2,3,4)
E3 = edge(P,'canny', [.04 .1], 3);
imshow(E3);
title('\sigma = 3');
subplot(2,3,5)
E4 = edge(P,'canny', [.04 .1], 4);
imshow(E4);
title('\sigma = 4');
subplot(2,3,6)
E5 = edge(P,'canny', [.04 .1], 5);
imshow(E5);
title('\sigma = 5');
\end{verbatim}

\includegraphics [width=4in]{Part_3_1_06.eps}
\begin{par}

The variable \texttt{tl} is the \emph{lower threshold}. The lower this
threshold is, the more details can be seen in the result. So the minimum
intensity change leading to a edge detection is controlled by \texttt{tl}.
Therefore every white point in a picture with lower threshold \texttt{tl}
is also white in every picture with a smaller lower threshold. This is
shown in by calculating the differences and finding their minimum and
maximum. There we also see see, that negative values for \texttt{lt}
produce the same output as $ \mathtt{lt} = 0 $.

\end{par} \vspace{1em}
\begin{verbatim}
subplot(2,3,1)
E_tl1 = edge(P,'canny', [ -.5 .1], 2);
imshow(E_tl1);
title('tl = -0.5');
subplot(2,3,2)
E_tl2 = edge(P,'canny', [.0 .1], 2);
imshow(E_tl2);
title('tl = 0');
subplot(2,3,3)
E_tl3 = edge(P,'canny', [.03 .1], 2);
imshow(E_tl3);
title('tl = 0.03');
subplot(2,3,4)
E_tl4 = edge(P,'canny', [.05 .1], 2);
imshow(E_tl4);
title('tl = 0.05');
subplot(2,3,5)
E_tl5 = edge(P,'canny', [.07 .1], 2);
imshow(E_tl5);
title('tl = 0.07');
subplot(2,3,6)
E_tl6 = edge(P, 'canny', [.0999 .1] , 2);
imshow(E_tl6);
title('tl = 0.0999');

%Calculate differences
diff1 = E_tl1 - E_tl2;
diff2 = E_tl2 - E_tl3;
diff3 = E_tl3 - E_tl4;
diff4 = E_tl4 - E_tl5;
diff5 = E_tl5 - E_tl6;

%max(diffx(:)) = 1 means white point in x but not in x+1
%min(diffx(:)) = 0 means no black point in x, which is white in x+1,
%otherwise the minimum would be -1

max1 = max(diff1(:));min1 = min(diff1(:));
max2 = max(diff2(:));min2 = min(diff2(:));
max3 = max(diff3(:));min3 = min(diff3(:));
max4 = max(diff4(:));min4 = min(diff4(:));
max5 = max(diff5(:));min5 = min(diff5(:));

if E_tl1 == E_tl2
    disp('Negative lower thresholds produce the same output as lt = 0');
end
if [max2 max3 max4 max5 ] == ones(1,4)
    disp('If lt increases, white points (noise or edges) disapear');
end
if [min2 min3 min4 min5] == zeros(1,4)
    disp('If lt increases, no black point turns white -> ')
    disp('number of detected points decreases');
end
\end{verbatim}

        \color{lightgray} \begin{verbatim}Negative lower thresholds produce the same output as lt = 0
If lt increases, white points (noise or edges) disapear
If lt increases, no black point turns white -> 
number of detected points decreases
\end{verbatim} \color{black}
    
\includegraphics [width=4in]{Part_3_1_07.eps}



% 	\end{document}
    
