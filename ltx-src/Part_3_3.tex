
% This LaTeX was auto-generated from an M-file by MATLAB.
% To make changes, update the M-file and republish this document.

%\documentclass{standalone}
%\usepackage{graphicx}
%\usepackage{color}
%\usepackage{standalone}�

%\sloppy
%\definecolor{lightgray}{gray}{0.5}
%\setlength{\parindent}{0pt}

%\begin{document}

    
    \begin{par}

We load the images and call the \texttt{disp\_map} method.

\end{par} \vspace{1em}
\begin{verbatim}
clear all
close all

PL = imread('images\corridorl.jpg');
PR = imread('images\corridorr.jpg');
PD = imread('images\corridor_disp.jpg');

disp_map = disp_map(PL,PR,11,11);
\end{verbatim}
\begin{par}

The result is visualized with the given command. The differences between
my result and the given solution are small, it is basically the same
picture. One problem is, that the maximum is not always unique. So
changing the search direction from \texttt{15:-1:0} to \texttt{0:15}
produces a different image.

\end{par} \vspace{1em}
\begin{verbatim}
figure
imshow(-disp_map,[-15 15]);
title('Is:');

figure;
imshow(PD);
title('Should be:');

figure;
imshow(PL);
\end{verbatim}

\includegraphics [width=4in]{Part_3_3_01.eps}

\includegraphics [width=4in]{Part_3_3_02.eps}

\includegraphics [width=4in]{Part_3_3_03.eps}



% 	\end{document}
    
