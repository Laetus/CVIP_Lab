
% This LaTeX was auto-generated from an M-file by MATLAB.
% To make changes, update the M-file and republish this document.

%\documentclass{standalone}
%\usepackage{graphicx}
%\usepackage{color}
%\usepackage{standalone}�

%\sloppy
%\definecolor{lightgray}{gray}{0.5}
%\setlength{\parindent}{0pt}

%\begin{document}

    
    \begin{par}

We load the images and call the \texttt{disp\_map} method. The
\texttt{tic, ..., toc,} command can be used to find out, how long the
statement between takes.

\end{par} \vspace{1em}
\begin{verbatim}
clear all
close all

PL = imread('images\corridorl.jpg');
PR = imread('images\corridorr.jpg');
PD = imread('images\corridor_disp.jpg');

tic,
disp_map = disp_map(PL,PR,11,11); toc,
\end{verbatim}

        \color{lightgray} \begin{verbatim}Elapsed time is 15.609133 seconds.
\end{verbatim} \color{black}
    \begin{par}

The result is visualized with the given command. It took about 19 seconds
to calculate the disparity map. On a 4 year old dual core CPU. So a state
of the art processor can do the task a little but quicker.

\end{par} \vspace{1em}
\begin{verbatim}
imshow(-disp_map,[-15 15]);
title('Solution:');

figure;
imshow(PD);
title('Should be:');

figure;
imshow(PL);
\end{verbatim}

\includegraphics [width=4in]{Part_3_3_01.eps}

\includegraphics [width=4in]{Part_3_3_02.eps}

\includegraphics [width=4in]{Part_3_3_03.eps}
\begin{par}

We see that the solution shows basically the same image. The small
differences are caused by the implementation. The maximum of the SSD must
not be unique. So the design of the argmax function can influence the
result. We see this clearly in the area associated with area in the back
of the corridor. We have a huge area, where all values are equal. So it
the SSD values are equal and the design of the argmax function has a huge
impact on the result. To show this, I copied the funtion
\texttt{disp\_map} to \texttt{disparity} and changed the search direction
in line 116 from \texttt{15:-1:0} to \texttt{0:15} and the result is in
this case better than the previous one.

\end{par} \vspace{1em}
\begin{verbatim}
disp_map2 = disparity(PL,PR,11,11);
imshow(-disp_map2,[-15,15]);
title('Solution 2');
\end{verbatim}

\includegraphics [width=4in]{Part_3_3_04.eps}



% 	\end{document}
    
