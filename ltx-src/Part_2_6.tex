
% This LaTeX was auto-generated from an M-file by MATLAB.
% To make changes, update the M-file and republish this document.

%\documentclass{standalone}
%\usepackage{graphicx}
%\usepackage{color}
%\usepackage{standalone} 

%\sloppy
%\definecolor{lightgray}{gray}{0.5}
%\setlength{\parindent}{0pt}

%\begin{document}

    
    \begin{par}

First we download the picture and display it.

\end{par} \vspace{1em}
\begin{verbatim}
close all;
clear;
P = imread('images/book.jpg');
imshow(P);
\end{verbatim}

\includegraphics [width=4in]{Part_2_6_01.eps}
\begin{par}

We use now \texttt{ginput} to find the coordinates of the 4 corners. We
start in the upper left corner and continue clockwise. We also try to
find the 4 corners of the book's title page. A DIN A4 paper has the
dimensions $210  mm \times 297$ mm. We fix the upper left corner to $(0,0)$.

\end{par} \vspace{1em}
\begin{verbatim}
%Applying ginput
[X_input,Y_input] = ginput(4);
% But we use the coordinates below, obtained from ginput, to get always the best
% result
points = [143,29;
          308,47;
          256,215;
          4,159;
          ];
% The desired coordinates are:
des =  [0,0;
        210,0;
        210,297;
        0,297;
        ];
% Here Y is the vertical and X the horizontal value.
X = points(:,1);
Y = points(:,2);
% Create the first 6 columns of A
A = [X,Y, ones(4,1), zeros(4,3); zeros(4,3), X,Y, ones(4,1)];
A_end = zeros(8,2);
%Compute the two last  columns
for i =1:4
    A_end(i,: ) = des(i,1) * points(i,:);
    A_end(i+4,:) = des(i,2) * points(i,:);
end
%Put everything together
A = [A,-A_end];
v = [des(:,1);des(:,2)];
\end{verbatim}
\begin{par}

Note that, the rows of $A$ and $v$ are permuted, compared to equation (*). We first consider the rows containing $x_{im}$ and then the
rows containing $y_{im}$. This trick makes it easier to create the matrix
in \texttt{MATLAB}. In linear algebra is shown, that this operation does not affect $u$ at all. The differences are shown below:

\end{par} \vspace{1em}
\begin{verbatim}
A
v
%Order rows
order = [1,5,2,6,3,7,4,8];
A_star = A(order,:)
v_star = v(order)
\end{verbatim}

        \color{lightgray} \begin{verbatim}
A =

  Columns 1 through 6

         143          29           1           0           0           0
         308          47           1           0           0           0
         256         215           1           0           0           0
           4         159           1           0           0           0
           0           0           0         143          29           1
           0           0           0         308          47           1
           0           0           0         256         215           1
           0           0           0           4         159           1

  Columns 7 through 8

           0           0
      -64680       -9870
      -53760      -45150
           0           0
           0           0
           0           0
      -76032      -63855
       -1188      -47223


v =

     0
   210
   210
     0
     0
     0
   297
   297


A_star =

  Columns 1 through 6

         143          29           1           0           0           0
           0           0           0         143          29           1
         308          47           1           0           0           0
           0           0           0         308          47           1
         256         215           1           0           0           0
           0           0           0         256         215           1
           4         159           1           0           0           0
           0           0           0           4         159           1

  Columns 7 through 8

           0           0
           0           0
      -64680       -9870
           0           0
      -53760      -45150
      -76032      -63855
           0           0
       -1188      -47223


v_star =

     0
     0
   210
     0
   210
   297
     0
   297

\end{verbatim} \color{black}
    \begin{par}

Now we we want to compute the variable $u = A^{-1} v$. To do that, we
use the \texttt{$\backslash$} operator

\end{par} \vspace{1em}
\begin{verbatim}
u = A\v
% Bringing it back into the original 3 times 3 matrix
U = reshape([u;1], 3, 3)';
%Testing
w = U*[X'; Y'; ones(1,4)];
w = w ./ (ones(3,1) * w(3,:));
%Is w and des close enough?
if sum(abs(w(1:2,:) -des')) < 1e-12
    T = maketform('projective', U');
    P1 = imtransform(P, T, 'XData', [0 210], 'YData', [0 297]);
    figure;
    imshow(P1);
else
    error('Projection did not work, use other points')
end
\end{verbatim}

        \color{lightgray} \begin{verbatim}
u =

    1.5198
    1.6250
 -264.4529
   -0.4211
    3.8598
  -51.7218
    0.0002
    0.0056

\end{verbatim} \color{black}
    
\includegraphics [width=4in]{Part_2_6_02.eps}
\begin{par}

The new image is a little bit blurred, especially in the upper half of the
picture. In the original picture, we see, that this is the part,
which is farer away from the camera. A pixel in the back of the original picture
represents a bigger area than a point in the foreground. Therefore we have
less information about the background and so the upper half of the new
picture must have worse quality.

\end{par} \vspace{1em}



% 	\end{document}
    
