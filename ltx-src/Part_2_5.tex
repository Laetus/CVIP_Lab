
% This LaTeX was auto-generated from an M-file by MATLAB.
% To make changes, update the M-file and republish this document.

%\documentclass{standalone}
%\usepackage{graphicx}
%\usepackage{color}
%\usepackage{standalone} 

%\sloppy
%\definecolor{lightgray}{gray}{0.5}
%\setlength{\parindent}{0pt}

%\begin{document}

    
    \begin{par}

In this exercise we use Fourier transformations to get rid of more
complex noise patterns. In this case, diagonal waves.

\end{par} \vspace{1em}
\begin{verbatim}
P = imread('images/pckint.jpg');
imshow(P);
\end{verbatim}

\includegraphics [width=4in]{Part_2_5_01.eps}
\begin{par}

Now we apply a Fourier Transformation, to see the picture in the frequency
domain. This transformation is very popular, so it is already implemented
in \texttt{MATLAB}.

\end{par} \vspace{1em}
\begin{verbatim}
F = fft2(P);
% F has complex values
whos F
S = abs(F);
% S has now real positive values
whos S
% Display shifted and transformed image
imagesc(fftshift(S.^0.1));
colormap('default')
\end{verbatim}

        \color{lightgray} \begin{verbatim}  Name        Size               Bytes  Class     Attributes

  F         256x256            1048576  double    complex   

  Name        Size              Bytes  Class     Attributes

  S         256x256            524288  double              

\end{verbatim} \color{black}
    
\includegraphics [width=4in]{Part_2_5_02.eps}
\begin{par}

Now we display the power spectrum without \texttt{fftshift} method. Then
the peaks are at the edges of the picture.help

\end{par} \vspace{1em}
\begin{verbatim}
imagesc(S.^0.1);
colormap('default')
% coord  = [up right; down left corner];
coord =    [ 17, 249;
            241, 9];
\end{verbatim}

\includegraphics [width=4in]{Part_2_5_03.eps}
\begin{par}

Now we set the area around the specified peaks to zero. This is done
"manually" to avoid a lot of error handling code.

\end{par} \vspace{1em}
\begin{verbatim}
F1 = F;
a = 2;
F1(17-a:17+a,249-a:249+a) = zeros(2*a + 1);
F1(241-a:241+a, 9-a:9+a) = zeros(2*a + 1);
S1 = abs(F1);
% Display shifted image
imagesc(fftshift(S1.^0.1));
colormap('default')
\end{verbatim}

\includegraphics [width=4in]{Part_2_5_04.eps}
\begin{par}

Now we transform $F_1$ back. To obtain the best result, we do here
contrast stretching again.

\end{par} \vspace{1em}
\begin{verbatim}
P1 = ifft2(F1);
imshow(P,[])
title('Before')
figure;
imshow(P1,[])
title('After')
\end{verbatim}

\includegraphics [width=4in]{Part_2_5_05.eps}

\includegraphics [width=4in]{Part_2_5_06.eps}
\begin{par}

This result is better, but if we look at the not shifted power spectrum,
we see, that the rows and columns in which the peaks are, are bigger,
than their neighbours. So in an additional step we set them also to zero.

\end{par} \vspace{1em}
\begin{verbatim}
F2 = F1;
% Get dimensions of the picture
[h w] = size(F1);
% sete rows and columns to zero
F2(:,9) = zeros(h,1);
F2(:,249) = zeros(h,1);
F2(17,:) = zeros(1,w);
F2(241,:) = zeros(1,w);
%Print power spectrum
imagesc(abs(F2).^.1);
colormap('default')
%Print Pictures
figure;
subplot(1,2,1)
imshow(P,[])
title('Before')
subplot(1,2,2)
imshow(P1,[])
title('After')
figure;
imshow(ifft2(F2),[])
title('After additional step')
\end{verbatim}

\includegraphics [width=4in]{Part_2_5_07.eps}

\includegraphics [width=4in]{Part_2_5_08.eps}

\includegraphics [width=4in]{Part_2_5_09.eps}



% 	\end{document}
    
