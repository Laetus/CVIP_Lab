
% This LaTeX was auto-generated from an M-file by MATLAB.
% To make changes, update the M-file and republish this document.

%\documentclass{standalone}
%\usepackage{graphicx}
%\usepackage{color}
%\usepackage{standalone} 

%\sloppy
%\definecolor{lightgray}{gray}{0.5}
%\setlength{\parindent}{0pt}

%\begin{document}

    
    \begin{par}

As we don't have to compute a additional matrix, the first step is to
load the picture again. To optimize the result, we use a contrast
stretching again.

\end{par} \vspace{1em}
\begin{verbatim}
P = imread('images/ntugn.jpg');
imshow(P,[]);
\end{verbatim}

\includegraphics [width=4in]{Part_2_4_01.eps}
\begin{par}

Now we apply the already implemented median filter with the
\texttt{medfilt2} method.

\end{par} \vspace{1em}
\begin{verbatim}
P3 = medfilt2(P); % as 3x3 is default
P5 = medfilt2(P, [5 5]);
imshow(P3,[]);
figure;
imshow(P5,[]);
\end{verbatim}

\includegraphics [width=4in]{Part_2_4_02.eps}

\includegraphics [width=4in]{Part_2_4_03.eps}
\begin{par}

We seem that the result get worse, if neighbourhood size increases. In
contrast to that, we see that in the ntusp picture the noise gets
cancelled completely.

\end{par} \vspace{1em}
\begin{verbatim}
Q = imread('images/ntusp.jpg');
imshow(Q,[]);
figure;
% Apply filters
Q3 = medfilt2(Q); % as 3x3 is default
Q5 = medfilt2(Q, [5 5]);
imshow(Q3,[]);
figure;
imshow(Q5,[]);
\end{verbatim}

\includegraphics [width=4in]{Part_2_4_04.eps}

\includegraphics [width=4in]{Part_2_4_05.eps}

\includegraphics [width=4in]{Part_2_4_06.eps}
\begin{par}

If the neighbourhood is too big, we lose again some details. So in this case,
the $3\times3$ filter has the best result. This is because the noise
points are always single pixels.

\end{par} \vspace{1em}



% 	\end{document}
    
