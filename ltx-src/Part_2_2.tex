
% This LaTeX was auto-generated from an M-file by MATLAB.
% To make changes, update the M-file and republish this document.

%\documentclass{standalone}
%\usepackage{graphicx}
%\usepackage{color}
%\usepackage{standalone} 

%\sloppy
%\definecolor{lightgray}{gray}{0.5}
%\setlength{\parindent}{0pt}

%\begin{document}

    
    \begin{par}

We display the image intensity histogram with 10 and 256 bins using the
following commands.

\end{par} \vspace{1em}
\begin{verbatim}
%Load P:
P = rgb2gray(imread('images/mrttrainbland.jpg'));
imhist(P,10);
figure
imhist(P,256);
\end{verbatim}

\includegraphics [width=4in]{Part_2_2_01.eps}

\includegraphics [width=4in]{Part_2_2_02.eps}
\begin{par}

We see that the histogram with 256 bins has a higher resolution. But the
first gives also a good summary. In both figures, we see that there are
more dark than bright pixels. We carry out the histogram equalization, with the following command.

\end{par} \vspace{1em}
\begin{verbatim}
P2 = histeq(P,255);
imshow(P2);
figure;
imhist(P2,256);
\end{verbatim}

\includegraphics [width=4in]{Part_2_2_03.eps}

\includegraphics [width=4in]{Part_2_2_04.eps}
\begin{par}

We see in the histogram and in the picture itself, that the number of
bright pixels increased. So we obtain a brighter picture. If we run the
equalisation again, nothing happens. To see that we subtract the two
pictures $P_2$ and $P_3$ and test if the result equals $0$. If that is the case, the
corresponding values of the pictures are equal and the value in the
result \texttt{test} is 1. The matrix \texttt{test} is a logical matrix,
that means that it only contains $1$ or $0$. So it the minimum of
\texttt{test} is 1, we know that every value must be $1$. Therefore the
pictures are equal.
of P

\end{par} \vspace{1em}
\begin{verbatim}
P3 = histeq(P2,255);
imshow(P3);
figure;
imhist(P3,256);
test = (imsubtract(P2,P3) == 0);
min(test(:))
\end{verbatim}

        \color{lightgray} \begin{verbatim}
ans =

     1

\end{verbatim} \color{black}
    
\includegraphics [width=4in]{Part_2_2_05.eps}

\includegraphics [width=4in]{Part_2_2_06.eps}



% 	\end{document}
    
