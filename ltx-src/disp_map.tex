
% This LaTeX was auto-generated from an M-file by MATLAB.
% To make changes, update the M-file and republish this document.

%\documentclass{standalone}
%\usepackage{graphicx}
%\usepackage{color}
%\usepackage{standalone}�

%\sloppy
%\definecolor{lightgray}{gray}{0.5}
%\setlength{\parindent}{0pt}

%\begin{document}

    
    \begin{verbatim}
function [ disp_map ] = disp_map( PL, PR, ny,nx )
\end{verbatim}
\begin{par}

At first, we convert calculate the image intensities. To do that we use
the \texttt{rgb2gray} command and apply the \texttt{double} method
afterwards. We initialize the parameter \texttt{disp\_map} which will
contain the disparity information.

\end{par} \vspace{1em}
\begin{verbatim}
    PL = double(rgb2gray(PL));
    PR = double(rgb2gray(PR));
    [sizey, sizex] = size(PL);
    disp_map = ones(sizey,sizex) * 1337;
\end{verbatim}
\begin{par}

Not every pixel in \texttt{PL} has a $n_y \times n_x$ neighbourhood. So
we enlarge \texttt{PL} with zeros, such that the pixels close to the
frame get a bigger neighbourhood

\end{par} \vspace{1em}
\begin{verbatim}
   enlPL = [ zeros(floor(ny/2),sizex) ; PL ; zeros(floor(ny/2),sizex) ];
   enlPL = [ zeros(sizey  + ny - 1 , floor(nx/2)), enlPL, zeros(sizey ...
            + ny -1, floor(nx/2))];
   PRsquare = PR.^2;
\end{verbatim}
\begin{par}

We have to calculate the disparity for every pixel. To do that, two loops
are used.

\end{par} \vspace{1em}
\begin{verbatim}
   for y = 1:sizey
        for x = 1:sizex
\end{verbatim}
\begin{par}

            Due to the preprocessing, the template is now easy to adress.
            The center of the template is the pixel y,x in \texttt{PL}

\end{par} \vspace{1em}
\begin{verbatim}
            template = enlPL(y:ny+y-1,x:nx+x-1);
\end{verbatim}
\begin{par}

            We are supposed to find the corresponding point in the same
            scanline. That means, that the y coordinates must be the
            same.  An other restriction is, that only disparities smaller
            than $15$ are considered. to save time, we calculate the SSD
            only for a small part of the image. The lower and upper
            bounds are calculated as follows. The image part has to be as big
            as the template, otherwise, we the SSD results are wrong. As
            we the corresponding point in \texttt{PR} must be less than 15 away, we have
            to find the maximum position in $31 = \underbrace{15}_{\text{left of} \ (y,x) } + \underbrace{15}_{\text{right of} \ (y,x) } + \underbrace{1}_{ the pixel (x,y)}$ pixels.
            
\end{par} \vspace{1em}
\begin{verbatim}
            ly = max(1,y - floor(ny/2));
            uy = min(sizey, y + floor(ny/2));
            lx = max(1, x - 15 - floor(nx/2));
            ux = min(sizex, x + 15 + floor(nx/2));
\end{verbatim}
\begin{par}

            The SSD is defined as the following \begin{align*} S(x,y) & =
            \sum \limits_{j=0}^{M} \sum
            \limits_{k=0}^{N} \left ( I(x+j,y+k)  - T(j,k) \right)^2  \\ & = \sum \limits_{j=0}^{M} \sum
            \limits_{k=0}^{N} I^2 (x+j, y+k) *1 + \underbrace{\sum \limits_{j=0}^{M} \sum
            \limits_{k=0}^{N} T^2(j,k)}_{= \ \text{constant}} - 2 \sum \limits_{j=0}^{M} \sum
            \limits_{k=0}^{N} I(x+j,y +k )T(j,k) \end{align*}
            If the first and third term had $(x-j,y-k)$ instead of
            $(x+j,y+k)$ we could write this as: \\ $$ I \oplus 2 - 2*
            I \oplus T $$, where $\oplus $ denotes convolution. But
            as we learned also correlation, here denoted with
            $\otimes $, and this is applied here. As the minimum must
            be found, the constant term can be omitted. So we get $$ I^2
            \otimes \mathds{1} - 2 (I \otimes T)$$ . \texttt{MATLAB}
            only provides the convolution method \texttt{conv2}. $\mathds{1}$ represents a matrix, which has the same size as $T$ and every entry is $1$.  But in
            the lecture was shown, that $$A \otimes B =  A \oplus
            rot180(B) $$. Multiplication with $-1$ turns the problem from
            a minimization to a maximization problem and we get: \\ $$ 2
            * (I \oplus
            \underbrace{\mathrm{rot}90(\mathrm{rot}90(T)}_{90+90 = 180} -
            I^2 \oplus \underbrace{\mathds{1}}_{\mathrm{rot}180(\mathds{1} ) = \mathds{1}} $$

\end{par} \vspace{1em}
\begin{verbatim}
            SSD =2*conv2(PR(ly:uy,lx:ux),rot90(rot90(template)),'same')...
                - conv2(PRsquare(ly:uy,lx:ux), ones(ny,nx),'same') ;
\end{verbatim}
\begin{par}

            As we cropped the image, we have to find the scanline and the
            position of x.

\end{par} \vspace{1em}
\begin{verbatim}
            %In which line in SSD is line y now?
            if ly ~= 1
                line = SSD(floor(ny/2) + 1,:);
            else
                line = SSD(end - floor(ny/2),:);
            end

            %Which entry of line is associated with x?
            if lx ~= 1
                xposition = floor(nx/2) + 15 +1;
            else
                xposition = length(line) - floor(nx/2) - 15 ;
            end
\end{verbatim}
\begin{par}

            We turned the original minimization problem into a
            maximization problem. Here we look for the disparity, which
            has the maximum SSD value and is smaller or equal to $15$.

\end{par} \vspace{1em}
\begin{verbatim}
            argmax = -Inf;
            disparity = -16;

            for t = 15:-1:0
                plus = xposition + t;
                minus = xposition - t;
                if  minus > 0 && line(minus) > argmax
                    argmax = line(minus);
                    disparity = -t;
                end
                if  plus <= length(line) && line(plus) > argmax
                    argmax = line(plus);
                    disparity = t;
                end
            end
\end{verbatim}
\begin{par}

            Finally we found the disparity for (y,x) and we add this
            value to the disparity matrix \texttt{disp\_map}. This function returns this parameter.

\end{par} \vspace{1em}
\begin{verbatim}
            disp_map(y,x) = disparity;
\end{verbatim}
\begin{verbatim}
        end
    end
\end{verbatim}
\begin{verbatim}
 end
\end{verbatim}



% 	\end{document}
    
